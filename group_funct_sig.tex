\documentclass[11pt]{llncs}
\usepackage{fullpage}

%
\usepackage{latexsym,color}
\usepackage{amscd,amsmath,amssymb}
\usepackage{amsfonts}
\usepackage{pifont}
\usepackage{paralist}
\usepackage{soul}
\usepackage{boxedminipage}
\usepackage{xspace}
\usepackage{dashbox}
\usepackage{graphicx}
\usepackage{hyperref}



\title{Notes on Malleable Signatures}

%\date{} % delete this line to display the current date
% \institute{
%IMDEA Software Institute, Madrid, Spain
%\\ \email{\{dario.fiore\}@imdea.org}
%\and
%ENS, CNRS, INRIA, and PSL, Paris, France
%\\ \email{Anca.Nitulescu@ens.fr}
% }

\author{}
\institute{}


\usepackage{color}
\usepackage{notes}
\usepackage{framed}

 \usepackage{a4wide}
 
 \usepackage{enumitem}
\usepackage{amssymb}
  % \setlength{\parskip}{0.1\baselineskip}
\usepackage{multicol}

%
\usepackage[utf8]{inputenc} %
%
\usepackage{amsmath,amssymb,booktabs}


\begin{document}
\maketitle
\input{header1}



%\section*{Functional Signatures}
%
%\subsubsection{Definition of Functional Signatures.} \cite{PKC:BoyGolIva14}.
%
%
%A functional signature scheme $\FSig$ for a message space $\M$ and function family 
%$\F =\lbrace f:\D_f \to \M \rbrace$  is a tuple of probabilistic, polynomial-time algorithms
%$(\FSetup, \FKG, \FSign, \FVer)$ that work as follows
%\begin{itemize}
%\item $\FSetup(1^\lambda)$: takes a security parameter $\lambda$ and outputs a 
%master verification key $\mvk$ and a master secret key $\msk$.
%
%\item $\FKG(\msk, f)$: takes the master secret key $\msk$ and a function $f \in \F$
%(represented as a circuit) and it outputs a signing key $\sk_f$ for $f$.
%
%\item $\FSign(\mvk, f, \sk_f, m)$: takes as input a function $f \in \F$, the signing key
%$\sk_f$ for that, and a message $m \in  \D_f$, and it outputs $(f(m), \sigma)$ where
%$\sigma$ represents a  signature on $f(m)$.
%
%\item $\FVer(\mvk,  m^*, \sigma)$: takes as input the master verification key $\mvk$,
%a message $m^* \in \M$ and a signature $\sigma$, and outputs either $1$ (accept)
%or $0$ (reject).
%\end{itemize}
%and satisfy {\em correctness}, {\em unforgeability},  and {\em function privacy}
%as described below. 
%\begin{itemize}
%
%\item \textbf{Correctness.} A functional signature scheme is correct if the following 
%holds with probability 1:
%$$\forall f \in \F, \ \forall m \in  \D_f, \ (\msk, \mvk) \gets \FSetup(1^\lambda), \  \sk_f  \gets \FKG(\msk, f), $$
%$$  (m^*,\sigma) \gets  \FSign(\mvk, f,  \sk_f, m), \FVer(\mvk, m^*, \sigma) = 1 $$
%
%
%
%\item \textbf{Unforgeablity.} A functional signature scheme is unforgeable if for 
%every PPT adversary $\A$ there is a negligible function $\epsilon$ such that 
%$\Pr[\fsufcma_{\A, \FSig}(\lambda) = 1] \leq \epsilon(\lambda)$ where the experiment 
%$\fsufcma_{\A, \FSig}(\lambda)$ is described in the following:\\
%
%\begin{itemize}
%
%\item[{ Key gen:}] The challenger generates $(\msk, \mvk) \gets  \FSetup(1^\lambda)$,
%and gives $\mvk$ to $\A$.
%
%\item[{ Queries:}] The adversary is allowed to adaptively query a key generation oracle $\Ok$
%and a signing oracle $\Os$, that share a dictionary $D$ indexed by tuples $(f, i) \in \F \times \mathbb{N}$,
%whose entries are signing keys. For answering these queries, the challenger proceeds as follows:
%
%\item $\Ok$ $(f, i)$:
%
%\begin{itemize}
%
%\item If $(f, i) \in D$ (i.e., the adversary had already
%queried the tuple $(f,i)$), then the challenger replies with the same
%key $\sk_f^i$ generated before.
%
%\item Otherwise, generate a new $\sk_f^i \gets \FKG(\msk,f)$, add the entry 
%$(f,i) \to \sk_f^i$ in $D$, and return $\sk_f^i$.
% 
%\end{itemize}
%
%\item $\Os$ $(f, i,m)$: 
%\begin{itemize}
%\item If there exists an entry for the key $(f, i)$ in $D$, then the challenger
%generates a signature on $f(m)$ using this key, i.e., 
%$\sigma \gets \FSign(\mvk, f, \sk_f^i, m)$.
%
%\item Otherwise, generate a new key $\sk_f^i \gets \FKG(\msk,f)$, add an
%entry $(f,i) \to \sk_f^i$ to $D$, and generate a signature on $f(m)$ using this
%key, i.e., $\sigma \gets \FSign(\mvk, f, \sk_f^i, m)$.\\
%
%\end{itemize}
%
%\item[Forgery:] After the adversary is done with its queries, it outputs a pair
%$(m^*, \sigma)$, and the experiment outputs 1 iff the following conditions hold
%\begin{itemize}
%\item $\FVer(\mvk,  m^*, \sigma)=1$.
%\item there does not exist $m$ such that $m^* = f(m)$ for any $f$ which was
% sent as a query to the $\Ok$ oracle.
%\item there does not exist a pair $(f,m)$ such that $(f,m)$ was a query to the
% $\Os$ oracle and $m^*= f(m)$.
%\end{itemize}
%\end{itemize}
%
%
%\item \textbf{Function privacy.} Intuitively, function privacy requires that the
%distribution of signatures on a message $m$ that are generated via different
%keys $\sk_f$ should be computationally indistinguishable, even given the 
%secret keys and master signing key. 
%More formally, a functional signature scheme has function privacy if for every
%PPT adversary $\A$ there is a negligible function $\nu$ such that
%$\Pr[\fspriv_{\A, \FSig}(\lambda) = 1] \leq \nu(\lambda)$ where experiment 
%$\fspriv_{\A, \FSig}(\lambda)$ works as follows:
%\begin{itemize}
%\item The challenger generates a key pair $(\mvk, \msk)  \gets \FSetup(1^\lambda)$
%and gives $(\mvk, \msk)$ to $\A$.
%\item The adversary chooses a function $f_0$ and receives an (honestly generated)
% secret key $\sk_{f_0} \gets \FKG(\msk, f_0)$.
%\item The adversary chooses a second function $f_1$ such that $|f_0|=|f_1|$ (where
% padding can be used if there is a known upper bound) and receives an (honestly 
% generated) secret key $\sk_{f_1} \gets \FKG(\msk, f_1)$.
%\item The adversary chooses a pair of values $(m_0, m_1)$ such that $|m_0| = |m_1|$
% and $f_0(m_0) = f_1(m_1)$.
%\item The challenger selects a random bit $b \gets \bits$ and computes a signature  
% on the image message $m^* =f_0(m_0) = f_1(m_1)$ using secret key $\sk_{f_b}$,
% and gives the resulting signature $\sigma \gets  \FSign(\sk_{f_b}, m_b)$ to $\A$.
%\item The adversary outputs a bit $b'$, and the experiment outputs 1 iff $b' = b$.
%\end{itemize} 
%\end{itemize}
%
%\textbf{Succinct Functional Signatures}
%
%A functional signature scheme is called {\em succinct} if there exists a polynomial
%$s(\cdot)$ such that, for every security parameter $\lambda \in \NN$, $f \in \F$,
%$m \in \D_{f}$, it holds with probability 1 over $(\mvk, \msk)  \gets \FSetup(1^\lambda)$,
%$\sk_{f} \gets \FKG(\msk, f)$, $(f(m), \sigma) \gets  \FSign(\sk_{f}, m)$ that
%$|\sigma| \leq s(\secpar, |f(m)|)$. In particular, the size of $\sigma$ is independent
%of the function's size, $|f|$, and the function's input size, $|m|$.
%
%
%\newpage
%
%\subsubsection{Non-Succinct Functional Signatures Construction}\label{sec:funcsigconstr}
%To build the scheme, we use two $\ufcma$-secure signature schemes,
%$\Sigmaf=(\fSignKG, \fSign, \fSignVer)$ for message space $\D$ and 
%$\Sigma_0=(\SignKG_0, \Sign_0, \SignVer_0)$ for message space $\M$.
%%together with a fully succinct zero-knowledge \OSNARK $\Pi=(\Gen, \Prove, \Ver)$ 
%%for the following $\NP$ language.
%%we obtain an unforgeable functional signature scheme satisfying succinctness and function privacy $\mathsf{FS}= (\FSetup,\FKG, \FSign, \FVer)$.
%
%%
%
%
%\headingb{The construction.} 
%Considering the signature schemes $\Sigma_0,\Sigma'$, we construct the functional signature 
%scheme $\FSig[\Sigma_0,\Sigma_1]= (\FSetup, \allowbreak \FKG, \allowbreak  \FSign, \allowbreak  \FVer)$
%as follows:
%\begin{itemize}
%\item $\FSetup(1^\lambda):$ \ \\
%Generate a pair of keys for $\Sigma_0$: $(\msk, \mvk) \gets \SignKG_0(1^{\secpar})$.\\
%Output $(\msk, \mvk)$. \\
%
%\item $\FKG(\msk, f):$ \ \\
%Generate a new key pair for the scheme $\Sigma'$: $(\fsk, \fvk)  
%\gets \fSignKG(1^{\secpar})$.\\
%Compute $\sigma_{\fvk} \gets  \Sign_0(\msk, f|\fvk)$, and let the certificate
%$c$ be $c = (f, \fvk, \sigma_{\fvk})$.\\
%Output $\sk_f = (\fsk, c)$.\\
%
%\item $\FSign(\mvk, f, \sk_f, m):$ \ \\
%Parse $\sk_f$ as $(\fsk, c = (f, \fvk, \sigma_{\fvk}))$. \\
%Sign $m$ using $\fsk$ in $\Sigma'$: $\sigma_m \gets  \fSign(\fsk,m)$.\\
%Output $\sigma=(m, f, \fvk, \sigma_{\fvk}, \sigma_m)$.\\
%
%
%\item $\FVer(\mvk,  m^*, \sigma) $:  \ \\
%Parse $\sigma=(m, f, \fvk, \sigma_{\fvk}, \sigma_m)$\\
%Check:
%$(1)  \  m^* = f(m), \hspace{9pt} (2) \ \fSignVer(\fvk, m, \sigma_m) = 1, \hspace{9pt} (3) \ \SignVer_0(\mvk, f|\fvk, \sigma_{\fvk}) = 1.$\\
% Output $0$ or $1$ accordingly. 
%% i.e., verify that $\pi$ is a valid argument of knowledge of a witness  $\sigma = (m, c = (f, \fvk, \sigma_{\fvk}), \sigma_m)$. \\
%
%\end{itemize}
%
%
%
%\subsubsection{Definition of Group Signatures.} \cite{RSA:BelShiZha05}.
%
%
%
%
%A group signature scheme $\GSig$ for a message space $\M$ and group of $n$ users, each of them with a unique identity $i \in [n]$ is a tuple of probabilistic, polynomial time algorithms
% $\GSig = (\GKg, \UKg, \GSig, \GVf, \Open, \Judge)$ whose intended
%usage and functionality are as follows. 
%
%
%
%\begin{itemize}
%\item $\GKg(1^n)$: takes a security parameter the number of users to obtain a triple $(\gpk, \gmsk, \gsk[i])$.\\
%The group public key $\gpk$, whose possession enables signature verification, is made public. \\
%The opening key $\gmsk$ is provided to the group manager. 
%All users that are group members have a private signing key, denoted $\gsk[i]$.\\
%
%\item $\UKg(1^n)$ outputs a personal public and private key pair $(\upk[i], \usk[i])$.\\
% We assume that the values $\upk$ are public. (Meaning, anyone can obtain an authentic copy of the personal
%public key of any user. This might be implemented via a PKI.)\\
%
%\item $ \GSig(\gpk, \gsk[i], m)$: a group member, in possession of its signing key $\gsk[i]$, 
%can apply the group signing algorithm for a message $m \in \M$ to obtain a signature $\sigma$.\\
%
%\item $\GVf(\gpk,  m, \sigma)$: anyone in possession of the group public key $\gpk$ can run the deterministic group signature
%verification algorithm on inputs a message $m$, and a candidate signature $\sigma$ for $m$, to obtain
%a bit $0/1$. \\
%
%\item $\Open(\gpk, \gmsk, m, \sigma)$: the group manager, using his master key
%can apply the deterministic opening algorithm to
%a message $m$, and a valid signature $\sigma$ of $m$ under $\gpk$.\\
%The algorithm returns a pair $(i, \tau)$, where $i$ is the identity of the member who produced $\sigma$
% and $\tau$ is a proof of this claim that can be verified via the $\Judge$ algorithm.\\
% 
% \item $\Judge(\gpk, j , m, \sigma, \tau)$: it aims to check that $\tau$ is a proof that user $j$
%produced $\sigma$. We note that the judge will base its verification on the public key of user $j$.
%
%\end{itemize}
%
%\subsubsection{A Short Group Signature construction.} \cite{C:BonBoySha04}
%We give the idea for construction of the short group signature 
%scheme of Boneh and all.
%Starting from an interactive Zero-Knowledge protocol for SDH, call it $ \ZK $, 
%and from the linear encryption protocol 
%$\LE$ we define a group signature scheme $\GSig[\ZK, \LE]= (\GKg, \UKg, \GSig, \GVf, \Open)$.
%
%\headingb{Building Blocks}
%
%Let $\GG_1, \GG_2, \GG_T$ be three (multiplicative) cyclic groups of prime order $p$, 
%where possibly $\GG_1 = \GG_2$. Let $g_1$ be a generator of $\GG_1$ and $g_2$ a generator of $\GG_2$. 
%Consider the isomorphism $\psi$ from $\GG_2$ to $\GG_1$, with $\psi(g_2) = g_1$ and the bilinear map 
%$e : \GG_1 \times  \GG_2 \to \GG_T$ 
%
%
%\heading{Linear Encryption.} The Decision Linear problem gives rise to the Linear
%encryption $\LE$ scheme, a natural extension of ElGamal encryption. Unlike
%ElGamal encryption, Linear encryption can be secure even in groups where a
%DDH-deciding algorithm exists. In this scheme, a user’s public key is a triple
%of generators $u, v, h \in \GG_1$, his private key is the exponents $x, y \in \ZZ_p$ such that
%$u^x = v^y = h$. To encrypt a message $M \in  \GG_1$, choose random values $a, b \in \ZZ_p$, and
%output the triple $(u^a, v^b,m \dot h^{a+b})$. To recover the message from an encryption
%$(T_1, T_2, T_3)$, the user computes $T_3/(T_1^x T_2^y)$. By a natural extension of the proof
%of security of ElGamal, $\LE$ is semantically secure against a chosen-plaintext
%attack, assuming Decision-LA holds.
%
%\heading{Zero-Knowledge Protocol for SDH}
%We present the protocol for proving possession of a solution to
%an SDH problem. The public values are $g_1, u, v, h \in \GG_1$ and $g_2, w \in \GG_2$. Here
%$w = g_2^\gamma$ for some (secret) $\gamma \in  \ZZ_p$. The protocol proves possession of a pair
%$(A, x)$, where $A \in \GG_1$ and $x \in \ZZ_p$, such that $A^{x+\gamma} = g_1$. Such a pair satisfies
%$e(A,wg_2^x) = e(g_1, g_2)$. We use a standard generalization of Schnorr's protocol for
%proving knowledge of discrete logarithm in a group of prime order.
%
%The prover, Alice, selects exponents $\alpha, \beta \getsr \ZZ_p$ and computes a
%linear encryption $(T_1, T_2, T_3)$ of $A$:
%$$T_1 \gets u^\alpha  \ T_2 \gets v^\beta  \  T_3 \gets Ah^{\alpha + \beta}$$
%
%She also computes two helper values $\delta_1 \gets x^\alpha$ and $\delta_2 \gets x^\beta$.
%
%Alice and Bob then undertake a proof of knowledge of values $(\alpha, \beta, x, \delta_1, \delta_2) $
%satisfying the following five relations:
%
%$$ u^\alpha = T_1   \hspace{28pt}  v^\beta =T_2 $$
%$$ e (T_3, g_2)^x \cdot e(h,w)^{-\alpha-\beta} \cdot e(h, g_2)^{-\delta_1-\delta_2} = e(g_1, g_2)/e(T_3, w)$$
%$$T_1^xu^{-\delta_1}=1  \hspace{23pt} T_2^xv^{-\delta_2}=1$$
%
%This proof proceeds as follows:
%%
%\[
%\begin{array}{@{}@{}l@{}c@{}l@{}@{}}
% \text{ Prover} \ \mathcal{A} & & \text{Verifier} \ \mathcal{B} \\
%\toprule
% \textbf{Step (1)}&&\\
%\text{compute } \LE \text{ $T_i$ of } A \ \ & &\\ 
%\text{and blinding values } R_i \ \ &&\\
%
%& \xrightarrow{\textstyle (T1, T2, T3, R1, R2, R3, R4, R5) } \\
%
%&&\textbf{Step (2)}\\
%&&  \text{challenge } c \getsr \ZZ_p\\ 
%
%& \xleftarrow {\textstyle \  \ c \ \ } \\
%
%\textbf{Step (3)} &&\\
%\text{using $c$, compute} & &\\
%s_\alpha \ s_\beta \ s_x \ s_{\delta_1}  \ s_{\delta_2}  \\
%
% & \xrightarrow{\textstyle (s_{\alpha} , s_{\beta} , s_x , s_{\delta_1} , s_{\delta_2})   } \\
% 
% & & \textbf{Step (4)}\\
%   & &    \text{ verify received values using } T_i, R_i \\
%     & &    \text{ accept/reject} \\
%  
%\bottomrule
%\end{array}
%\]
%
%
%
%\headingb{GS Construction}
%\begin{itemize}
%
%\item $\GKg(1^n)$: 
%The group public key $\gpk=(g_1, g_2, h, u, v, w)$, whose possession enables signature verification, is made public. \\
%The opening key $\gmsk$ is the private key of the linear encryption $\xi_1, \xi_2$ (the exponents s.t. $u^{\xi_1}=v^{\xi_2}=h$).\\
%Each user private signing key $\gsk[i]=(A_i, x_i)$ is a solution to
%a SDH problem for $w = g^\gamma$.\\
%No party is allow to possess $\gamma$.\\
%
%
%\item $ \GSig(\gpk, \gsk[i], M)$: A signature $\sigma$ is a
%proof of knowledge, which is itself a transcript of the $\ZK$ protocol including 
%the linear encryption $(T_1, T_2, T_3)$ of $A$ for some $\alpha, \beta \in \ZZ_p$:
%$$T_1 \gets u^\alpha  \ T_2 \gets v^\beta  \  T_3 \gets Ah^{\alpha + \beta}$$
%
%\item $\GVf(\gpk,  M, \sigma)$: Verifying the signature entails verifying that the transcript is correct.\\
%
%\item $\Open(\gpk, \gmsk, m, \sigma)$: The group manager,first verifies that $\sigma$ is a valid signature on $m$.\\
%Then, using his master key $\gmsk = (\xi_1, \xi_2)$ it
%can decrypt $(T_1, T_2, T_3)$ to recover $A \gets T_3/(T_1^{\xi_1}T_2^{\xi_2})$.\\ 
%If the group manager is given the elements $\{A_i\}$ of the user's private keys, it can look up the user 
%index corresponding to the identity $A$ recovered from the signature and so, trace it back.
% \end{itemize}
%%
%




\subsection{Functional Group Signatures.} \cite{PKC:BoyGolIva14}.



\subsubsection{Definition \cite{RSA:BelShiZha05}.


A functional group signature scheme $\FGSig$ for a message space $\M$, a function family 
$\F =\lbrace f:\D_f \to \M \rbrace$ and group of $n$ users, each of them with a unique identity $i \in [n]$ is a tuple of probabilistic, polynomial time algorithms
 $\FGSig = (\FGSetup, \FGKG, \FGSign, \FGVer)(\FGKg, \FUKg, \FOpen, \FJudge)$ whose intended
usage and functionality are as follows:


\begin{itemize}
\item $\FGSetup(1^\lambda, n)$: takes a security parameter $\lambda$, the number $n$ of users to obtain a triple $(\gpk_f, \gmsk_f, \gsk_f[i])$ for each group and each function.\\ 
Outputs a master verification key $\mvk$ and a master secret key $\msk$.

\item $\FKG(\msk, f)$: takes the master secret key $\msk$ and a function $f \in \F$
(represented as a circuit) and it outputs a signing key $\sk_f$ for $f$.

\item $\FSign(\mvk, f, \sk_f, m)$: takes as input a function $f \in \F$, the signing key
$\sk_f$ for that, and a message $m \in  \D_f$, and it outputs $(f(m), \sigma)$ where
$\sigma$ represents a  signature on $f(m)$.

\item $\FVer(\mvk,  m^*, \sigma)$: takes as input the master verification key $\mvk$,
a message $m^* \in \M$ and a signature $\sigma$, and outputs either $1$ (accept)
or $0$ (reject).
\end{itemize}

\begin{itemize}
\item $\GKg(1^n)$: takes a security parameter 
The group public key $\gpk$, whose possession enables signature verification, is made public. \\
The opening key $\gmsk$ is provided to the group manager. 
All users that are group members have a private signing key, denoted $\gsk[i]$.\\

\item $\UKg(1^n)$ outputs a personal public and private key pair $(\upk[i], \usk[i])$.\\
 We assume that the values $\upk$ are public. (Meaning, anyone can obtain an authentic copy of the personal
public key of any user. This might be implemented via a PKI.)\\

\item $ \GSig(\gpk, \gsk[i], m)$: a group member, in possession of its signing key $\gsk[i]$, 
can apply the group signing algorithm for a message $m \in \M$ to obtain a signature $\sigma$.\\

\item $\GVf(\gpk,  m, \sigma)$: anyone in possession of the group public key $\gpk$ can run the deterministic group signature
verification algorithm on inputs a message $m$, and a candidate signature $\sigma$ for $m$, to obtain
a bit $0/1$. \\

\item $\Open(\gpk, \gmsk, m, \sigma)$: the group manager, using his master key
can apply the deterministic opening algorithm to
a message $m$, and a valid signature $\sigma$ of $m$ under $\gpk$.\\
The algorithm returns a pair $(i, \tau)$, where $i$ is the identity of the member who produced $\sigma$
 and $\tau$ is a proof of this claim that can be verified via the $\Judge$ algorithm.\\
 
 \item $\Judge(\gpk, j , m, \sigma, \tau)$: it aims to check that $\tau$ is a proof that user $j$
produced $\sigma$. We note that the judge will base its verification on the public key of user $j$.

\end{itemize}


\subsubsection*{Open questions}
\ \\

\heading{Outsourcing Setting}

How can we obtain privacy for  $f(m)$ in $\FSign$ algorithm. 

One possible way is to use FHE and hide the message $m$ by encrypting it under de FHE scheme  $c = FHE(m)$.

The $\FVer(\mvk,  FHE(f(c)), \sigma) $ needs now an additional procedure of extracting (decrypting) to get back the evaluation $f(m)$ from  $FHE(f(c))$: $Dec(FHE(f(c))=f(Dec(c))=f(m)$.

\ \\

\heading{Functional Group Signatures} 

How can we obtain a new primitive/protocol that achieves group signature (\cite{C:BonBoySha04}) and functional signature proprieties at the same time.

We want to associate different functions $f_i$ to different users of a group while preserving the fact that signatures coming from different users are indistinguishable.

The same type of construction can be used to obtain Group Signatures for Homomorphic Encryption.

 The issue can be the tracing algorithm required by the Group Signature setting. What would we like to trace in these new protocols?
 
 An interesting primitive to use would be the Constrain PRFs (or Key Homomorphic PRFs). 


 
\bibliographystyle{alpha}
\bibliography{abbrev3,crypto}  % sigproc.bib is the name of the Bibliography in this case
% You must have a proper ".bib" file
%  and remember to run:
% latex bibtex latex latex
% to resolve all references
%
% ACM needs 'a single self-contained file'!
%
%APPENDICES are optional
%\balancecolumns
\end{document}