\documentclass[11pt]{llncs}
\usepackage{fullpage}

%
\usepackage{latexsym,color}
\usepackage{amscd,amsmath,amssymb}
\usepackage{amsfonts}
\usepackage{pifont}
\usepackage{paralist}
\usepackage{soul}
\usepackage{boxedminipage}
\usepackage{xspace}
\usepackage{dashbox}
\usepackage{graphicx}
\usepackage{hyperref}



\title{Notes on Malleable Signatures}

%\date{} % delete this line to display the current date
% \institute{
%IMDEA Software Institute, Madrid, Spain
%\\ \email{\{dario.fiore\}@imdea.org}
%\and
%ENS, CNRS, INRIA, and PSL, Paris, France
%\\ \email{Anca.Nitulescu@ens.fr}
% }

\author{}
\institute{}


\usepackage{color}
\usepackage{notes}
\usepackage{framed}

 \usepackage{a4wide}
 
 \usepackage{enumitem}
\usepackage{amssymb}
  % \setlength{\parskip}{0.1\baselineskip}
\usepackage{multicol}

%
\usepackage[utf8]{inputenc} %
%
\usepackage{amsmath,amssymb,booktabs}


\begin{document}
\maketitle



\newtheorem{clm}{Claim}

% Games

\newenvironment{myexample}[1]{
  \medskip\noindent
  \begin{center}
  \ifFullVersion
    \begin{boxedminipage}{160mm}
  \else
    \begin{boxedminipage}{85mm}
  \fi
  \textbf{Example (#1).}
}{
  \end{boxedminipage}
  \end{center}
}



%useful commands
\newcommand{\ignore}[1]{}
\newcommand{\headingi}[1]{{\textsc{#1}\xspace}}
\newcommand{\heading}[1]{{\medskip\noindent\headingi{#1}~}}
\newcommand{\headingb}[1]{{\medskip\noindent\textbf{#1}\xspace}}
\newcommand{\myparagraph}[1]{{\medskip\noindent\textbf{#1}. }}
\renewcommand\paragraph[1]{\smallskip\textbf{#1.}}


\newcommand{\Exp}{\mathbf{Exp}}
\def\Adv{\mathbf{Adv}}


\newcommand{\BigFig}[1]{\parbox{12pt}{\Huge #1}}
\newcommand{\BigZero}{\BigFig{0}}

\newcommand{\set}[2]{\{#1, \ldots, #2\}}
\newcommand{\done}{\ensuremath{\texttt{*}}}

\def\get{{\leftarrow}}
\def\getsr{\stackrel{{\scriptscriptstyle\$}}{\leftarrow}}
\def\bits{\{0,1\}}
\newcommand{\myset}[2]{\{#1, \ldots,#2\}}

\def\one{\ensuremath{^{(1)}}}    % (1) in the exponent
\def\two{\ensuremath{^{(2)}}}    % (2) in the exponent
\def\arb{\ensuremath{^{(i)}}}    % (i) in the exponent

\def\yA{{\ensuremath{y_0}}}
\def\yB{\ensuremath{Y_1}}
\def\yC{\ensuremath{\hat Y_2}}

%%
\newcommand{\Gm}{{\sf G}}
\newcommand{\GmZK}{{\sf ZK\prime G}}
\def\Bad{{\sf Bad}}
\def\bad{{\sf bad}}
\def\Good{{\sf Good}}
\def\GoodGuess{{\sf GoodGuess}}
\def\QueryMatch{{\sf QueryMatch}}
\def\true{{\sf true}}
\def\false{{\sf false}}
\def\typei{{\sf Type\mbox{-}I}}
\def\typeii{{\sf Type\mbox{-}II}}
\def\Ev{{\sf Ev}}

\def\QueryGuess{{\sf MsgSample}}

\def\counter{{\sf Cnt}}

%symbols

\def\p{\mathsf{p}}

\def\PP{\mathsf{P}}



\def\A{{\cal A}}
\def\B{{\cal B}}
\def\C{{\cal C}}
\def\D{{\cal D}}
\def\E{{\cal E}}
\def\F{{\cal F}}
\def\G{{\cal G}}
\def\H{{\cal H}}
\def\I{{\cal I}}
\def\J{{\cal J}}
\def\K{{\cal K}}
\def\L{{\cal L}}
\def\M{{\cal M}}
\def\N{{\cal N}}
\def\O{{\cal O}}
\def\P{{\cal P}}
\def\R{{\cal R}}
\def\S{{\cal S}}
\def\T{{\cal T}}
\def\U{{\cal U}}
\def\V{{\cal V}}
\def\W{{\cal W}}
\def\X{{\cal X}}
\def\Y{{\cal Y}}
\def\Z{{\cal Z}}
\def\ZZ{\mathbb Z}
\def\qrn{\mathbb{QR}_N}
\def\QQ{\mathbb Q}
\def\FF{\mathbb F}
\def\GG{\mathbb G}
\def\RR{\mathbb R}
\def\NN{\mathbb N}
\def\Rho{\mathrm P}
\def\OO{\mathbb O}
\def\NP{{\sf NP}}


\def\vA{\vv a}
\def\vB{\vv b}
\def\vC{\vv c}

\def\sA{_{\sf a}}
\def\sB{_{\sf b}}
\def\sC{_{\sf c}}

\def\zzn{\mathbb{Z}^{*}_{N}}
\def\zp{\mathbb{Z}_{p}}
\def\rsagen{{\sf RSAGen}}

%Oracles:


\def\Os{\cal O_{\mathsf{sign}}}
\def\Ok{\cal O_{\mathsf{key}}}

\def\RO{{\sf RO}}

\def\Extr{\E}
\def\querytranscript{{\sf qt}}

\def\xor{\oplus}

\def\negl{{\sf negl}}
\def\poly{{\sf poly}}
\def\secpar{\lambda}


\def\pp{{\sf pp}}
\def\blin{{\sf bg}}
\def\blinpp{{\sf bgpp}}
\def\td{{\sf td}}
\def\prfpp{{\sf prfpp}}

\def\gate{{f_{g}}}
\def\gateadd{{f_{+}}}
\def\gatemult{{f_{\times}}}

%algebraic PRF
\def\prfkg{{\sf F.KG}}
\def\prf{{\sf F}}
\def\cfeval{{\sf CFEval}}
\newcommand{\cfevaloff}[1]{\ensuremath{{\sf CFEval^{off}_{#1}}}}
\newcommand{\cfevalon}[1]{\ensuremath{{\sf CFEval^{on}_{#1}}}}
\newcommand{\CFEvalOff}[1]{\ensuremath{{\sf CFEval^{off}_{#1}}}}
\newcommand{\CFEvalOn}[1]{\ensuremath{{\sf CFEval^{on}_{#1}}}}
\def\comp{{\sf Comp}}
\newcommand\zst{\ensuremath{\kappa^*\!}}
\newcommand\tst{\ensuremath{z^*\!}}
\newcommand\ip[3]{\ensuremath{\langle #1 , #2 \rangle_{#3}}}
\newcommand\ivp[3]{\ip{\vv{#1}}{\vv{#2}}{#3}}

\def\primitive{\adSNARK}
\def\aprimitive{an \adSNARK}
\def\Aprimitive{An \adSNARK}
\def\primitives{\adSNARKs}

\def\primitiveG{\adSNARG}
\def\aprimitiveG{an \adSNARG}
\def\primitivesG{\adSNARGs}

% System / Scheme
\def\ADSNARG{\ensuremath{{\sf ADSNARG}}\xspace}
\def\ADSNARK{\ensuremath{{\sf ADSNARK}}\xspace}
\def\PGHR{\ensuremath{{\sf PGHR}}\xspace}
\def\ADPGHR{\ensuremath{{\sf AD\mbox{-}PGHR}}\xspace}

% Primitive
\def\OSNARK{\mbox{O-SNARK}\xspace}
\def\OSNARKs{\mbox{O-SNARKs}\xspace}
\def\ZKOSNARK{\ensuremath{{\sf\mbox{ZK-AD-SNARK}}}\xspace}

\def\APoK{{\sf AdPoK}}
\def\OAPoK{{\sf O\mbox{-}AdPoK}}

\def\VCompAuth{{\sf VCompAuth}\xspace}
\def\VCompAuthPub{{\sf VCompAuthPub}\xspace}
\def\VCAD{{\sf VCAD}\xspace}

%homomorphic signature
\def\HomSig{\ensuremath{{\sf HomSig}}\xspace}

\def\HomKG{{\sf HomKG}}
\def\HomSign{{\sf HomSign}}
\def\HomVer{{\sf HomVer}}
\def\HomEval{{\sf HomEval}}
\def\Hide{{\sf Hide}}
\def\HideVer{{\sf HideVer}}


%funtional signature
%group signature
%other schemes

\def\FSig{\ensuremath{{\sf FS}}\xspace}
\def\GSig{\ensuremath{{\sf GS}}\xspace}
\def\PSig{\ensuremath{{\sf PS}}\xspace}


\def\LE{\ensuremath{{\sf LE}}\xspace}
\def\ZK{\ensuremath{{\sf ZK}}\xspace}

\def\FSetup{{\sf FS.Setup}}
\def\FKG{{\sf FS.KeyGen}}
\def\FSign{{\sf FS.Sign}}
\def\FVer{{\sf FS.Ver}}
\def\Afs{\A_{\mathsf{FS}}}
\def\Apr{\A_{\mathsf{priv}}}


\def\PSetup{{\sf P.Setup}}
\def\PSign{{\sf P.Sign}}
\def\Derive{{\sf P.Derive}}
\def\PVer{{\sf P.Ver}}


\def\GKg{{\sf GS.Setup}}
\def\UKg{{\sf GS.KeyGen}}
\def\Join{{\sf GS.Join}}
\def\Iss{{\sf GS.Iss}}
\def\GSig{{\sf GS.Sign}}
\def\GVf{{\sf GS.Ver}}
\def\Open{{\sf GS.Open}}
\def\Judge{{\sf GS.Judge}}
\def\Ags{\A_{\mathsf{GS}}}


\def\win{\mathsf{win}}

\def\multisig{\ensuremath{{\sf m\Sigma}}}



% Algorithms
\def\Setup{{\sf Setup}}
\def\KeyGen{{\sf KeyGen}}
\def\FuncPrep{{\sf FuncPrep}}
\def\Gen{{\sf Gen}}
\def\QueryPrep{{\sf QueryPrep}}
\def\Merge{{\sf Merge}}
\def\Auth{{\sf Auth}}
\def\AuthKG{{\sf AuthKG}}
\def\AuthVer{{\sf AuthVer}}
\def\Comp{{\sf Comp}}
\def\Prove{{\sf Prove}}
\def\ReRand{{\sf ReRand}}
\def\Ver{{\sf Ver}}
\def\rhoMerge{\ensuremath{K}}

%Signature Oracle
\def\sign{{\sf Sign}}

%signatures
\def\hatS{\widehat{\Sigma}}
\def\tildeS{\widetilde{\Sigma}}
\def\hats{\hat{\sigma}}
\def\tildes{\tilde{\sigma}}
\def\bars{\bar{\sigma}}

\def\SignKG{{\sf kg}}
\def\Sign{{\sf sign}}
\def\SignVer{{\sf vfy}}
\def\ufcma{{\sf UF\mbox{-}CMA}}
\def\sufcma{{\sf SUF\mbox{-}CMA}}

\def\ZKKeyGen{{\sf ZK\prime KeyGen}}
\def\ZKComp{{\sf ZK\prime Comp}}
\def\ZKFuncPrep{{\sf ZK\prime FuncPrep}}
\def\ZKVer{{\sf ZK\prime Ver}}

\def\VerPrep{{\sf VerPrep}}
\def\EffVer{{\sf EffVer}}
\def\Eval{{\sf Eval}}
\def\GateEval{{\sf GateEval}}
\def\vgeval{{\sf VerGateEval}}
\def\VerGateEval{{\sf VerGateEval}}
\def\Sim{{\sf Sim}}
\def\rplus{\fbox{$\stackrel{{+}}{\rightarrow}$}}
\def\lplus{\fbox{$\stackrel{{+}}{\leftarrow}$}}


\def\Sigmaf{\Sigma'}
\def\fSignKG{{\sf kg}'}
\def\fSign{{\sf sign}'}
\def\fSignVer{{\sf vfy}'}

% Keys
\def\sk{{\sf sk}}
\def\pk{{\sf pk}}
\def\vk{{\sf vk}}
\def\pap{{\sf pap}}
\def\SK{{\sf SK}}
\def\PK{{\sf PK}}
\def\VK{{\sf VK}}


\def\gmsk{\sf gmsk} %open key
\def\gpk{\sf gpk}
\def\gsk{\sf gsk}
\def\upk{\sf upk}
\def\usk{\sf usk}

\def\hsk{\widehat{\sk}}
\def\tsk{\widetilde{\sk}}
\def\hvk{\widehat{\vk}}
\def\tvk{\widetilde{\vk}}

\def\MSK{{\sf MSK}}
\def\MVK{{\sf MVK}}

\def\msk{{\sf msk}}
\def\mvk{{\sf mvk}}

\def\mainsk{{\sf msk}_0}
\def\mainvk{{\sf mvk}_0}
\def\fsk{{\sf sk}'}
\def\fvk{{\sf vk}'}



\newcommand\QVK[2]{{\sf QVK}_{#1/#2}}
\def\EK{{\sf EK}}
\def\crs{{\sf crs}}
\def\prs{{\sf prs}}
\def\vst{{\sf vst}}
\def\trap{{\sf tr}}
\def\aux{\ensuremath{\mathit{aux}}\xspace}



\def\id{{\it id}}   % identity function
\def\did{{ \Delta}}
\def\iid{{\lab}}
\def\dset{{\sf D}}
\def\authtag{{\sigma}}
\def\authtagx{\authtag}
%\def\authtagx{\authtag^{\sf x}}
%\def\authtagw{\authtag^{\sf w}}
\def\GroupEval{{\sf GroupEval}}
\def\PolyEval{{\sf PolyEval}}

\def\fhe{{\sf FHE}}
\def\fheEval{{\sf FHE.Eval}}
\def\fheKeyGen{{\sf FHE.KeyGen}}
\def\fheEnc{{\sf FHE.Enc}}
\def\fheDec{{\sf FHE.Dec}}

\def\Poly{P}

\def\hufcma{\Exp^{\sf HomSig\mbox{-}UF}}
\def\fsufcma{\Exp^{\sf FS\mbox{-}UF}}
\def\fspriv{\Exp^{\sf FS\mbox{-}FPri}}
\def\zk{\Exp^{\Pi{\sf \mbox{-}ZK}}_{\A_0}}

% Labels, Multi-Labels, etc...
\def\ml{multi-label\xspace}
\def\mls{\ml{s}\xspace}
\def\mled{\ml{ed}\xspace}
\def\ML{Multi-Label\xspace}
\def\MLed{\ML{ed}\xspace}
\def\Ml{Multi-label\xspace}
\def\Mled{\Ml{ed}\xspace}
\def\lab{{\tau}}
\def\mlab{{\sf L}}
%\def\mlabx{\mlab^{\!\sf x}}
%\def\mlabw{\mlab^{\!\sf w}}
\def\mlabs{{\cal L}}

% QAP
\def\cin{\mathit{in}}
\def\cst{\mathit{x}}
\def\cout{\mathit{out}}
\def\cmid{\mathit{mid}}
\def\cinmid{\cin+\cmid}
\def\cauth{\mathit{\sigma}}
\def\cunauth{\mathit{\star}}
\def\QAP{\ensuremath{\mathrm{QAP}}\xspace}
\def\QAPInst{\ensuremath{\mathsf{QAPInst}}}
\def\QAPwit{\ensuremath{\mathsf{QAPwit}}}


% Closed-form efficiency
\def\cf{closed-form\xspace}
\def\CF{Closed-Form\xspace}

\def\NotZero{{\tt NotZero}}


%TwoMultMAC
\def\Addo{{\sf Add_0}}
\def\AddI{{\sf Add_1}}
\def\AddII{{\sf Add_2}}
\def\Multo{{\sf Mult_0}}
\def\MultI{{\sf Mult_1}}
\def\CMultI{{\sf ConstMult_1}}
\def\CMultII{{\sf ConstMult_2}}
\def\ShiftI{{\sf Shift_{0 \rightarrow 1}}}
\def\ShiftII{{\sf Shift_{1 \rightarrow 2}}}


% ===================================================================
\newcommand{\kwfont}[1]{{\ensuremath{\mathtt{#1}}}}
\newcommand{\kwfunction}{{\kwfont{function}\ }}
\newcommand{\kwlabel}{{\kwfont{label}\ }}
\newcommand{\kwfor}{{\kwfont{for}\ }}
\newcommand{\kwto}{{\kwfont{to}\ }}
\newcommand{\kwand}{~\kwfont{and}~}
\newcommand{\kwor}{{\kwfont{or}\ }}
\newcommand{\kwnot}{{\kwfont{not}}}
\newcommand{\kwdo}{{\kwfont{do}\ }}
\newcommand{\kwreturn}{{\kwfont{return}\ }}
\newcommand{\kwReturn}{{\kwfont{Return}\ }}
\newcommand{\kwalgorithm}{{\ensuremath{\mathbf{Algorithm}\ }}}
\newcommand{\kwprotocol}{{\kwfont{Protocol}\ }}
\newcommand{\kwexperiment}{{\kwfont{Experiment}\ }}
\newcommand{\kwadversary}{{\kwfont{Adversary}\ }}
\newcommand{\kworacle}{{\kwfont{Oracle}\ }}
\newcommand{\kwuntil}{{\kwfont{until}\ }}
\newcommand{\kwrepeat}{{\kwfont{repeat}\ }}
\newcommand{\kwif}{{\kwfont{if}\ }}
\newcommand{\kwthen}{{\kwfont{then}\ }}
\newcommand{\kwelse}{{\kwfont{else}\ }}
\newcommand{\kwwith}{{\kwfont{with}\ }}
\newcommand{\kwabort}{{\kwfont{abort}\ }}
\newcommand{\kwgoto}{{\kwfont{goto}\ }}
\newcommand{\kwwhile}{{\kwfont{while}\ }}
\newcommand{\kwparse}{{\kwfont{parse}\ }}
\newcommand{\kwas}{{\kwfont{as}\ }}
\newcommand{\kwstatic}{{\kwfont{static}\ }}
\newcommand{\kwrun}{{\kwfont{run}\ }}
\newcommand{\kwbegin}{{\kwfont{begin}\ }}
\newcommand{\kwend}{{\kwfont{end}\ }}
\newcommand{\kwstart}{{\kwfont{start}\ }}
\newcommand{\kwcontinue}{{\kwfont{continue}\ }}
\newcommand{\kwdefine}{{\kwfont{define}\ }}
\newcommand{\kwflip}{{\kwfont{flip}\ }}
\newcommand{\kwlet}{{\kwfont{let}\ }}
\newcommand{\kwof}{{\kwfont{of}\ }}
\newcommand{\kwcase}{{\kwfont{case}\ }}
\newcommand{\kwswitch}{{\kwfont{switch}\ }}
\newcommand{\kwpick}{{\kwfont{pick}\ }}
\newcommand{\kwfetch}{{\kwfont{fetch}\ }}
\newcommand{\kwfrom}{{\kwfont{from}\ }}
\newcommand{\kwset}{{\kwfont{set}\ }}
\newcommand{\kwcompute}{{\kwfont{compute}\ }}
\newcommand{\comment}[1]{\hspace{15pt}{\small /$\!\!$/\ #1}}
\newcommand{\Comment}[1]{\hspace{5pt}{/$\!\!$/\ #1}}
\newcommand\veq[1]{($V_{#1}$)\xspace}
\newcommand\veqA[1]{($A.{#1}$)\xspace}
\newcommand\veqP[1]{($P.{#1}$)\xspace}


%% Game-related macros
% There is a counter for the games.
\newcounter{gm}
\newcommand\setgame[1]{\setcounter{gm}{#1}}
% Print line number w/o game prefix.
\newcommand\fn{\footnotesize}
% Print line number w/ game prefix.
\newcommand\fng{\fn\thegm}
%
\newcommand\listQ{{\sf Q}}
\newcommand\listS{{\sf S}}
\newcommand\listT{{\sf T}}
\newcommand\Span{\mathit{Span}}
\newcommand\List{\mathit{List}}
\newcommand\kwcheck{\ensuremath{\mathit{check}}}
\newcommand{\kwout}{{\kwfont{GameOutput}}}
\newcommand{\procfont}[1]{\textbf{{#1}}}
\newcommand{\tablefont}[1]{\mathsf{#1}}
\newcommand{\procInitialize}{\procfont{Initialize}}
\newcommand{\InitializeCode}{\underline{\textbf{procedure} \procfont{Initialize}}}
\newcommand{\procFinalize}{\procfont{Finalize}}
\newcommand{\FinalizeCode}{\underline{\textbf{procedure} \procfont{Finalize}}}
\newcommand{\procGen}{{\procfont{Gen}}}
\newcommand{\GenCode}[1]{\underline{\textbf{procedure} \procfont{Gen}$(#1)$}}
\newcommand{\procFuncPrep}{{\procfont{FuncPrep}}}
\newcommand{\FuncPrepCode}[1]{\underline{\textbf{procedure} \procfont{FuncPrep}$(#1)$}}
\newcommand{\procQueryPrep}{{\procfont{QueryPrep}}}
\newcommand{\QueryPrepCode}[1]{\underline{\textbf{procedure} \procfont{QueryPrep}$(#1)$}}
\newcommand{\MergeCode}[1]{\underline{\textbf{procedure} \procfont{Merge}$(#1)$}}
\newcommand{\procAuth}{{\procfont{Auth}}}
\newcommand{\AuthCode}[1]{\underline{\textbf{procedure} \procfont{Auth}$(#1)$}}
\newcommand{\procVer}{{\procfont{Ver}}}
%\newcommand{\procVerE}{{\procfont{Ver}}_E}
\newcommand{\VerCode}[1]{\underline{\textbf{procedure} \procfont{Ver}$(#1)$}}
%\newcommand{\VerECode}[1]{\underline{\textbf{procedure} \procfont{Ver}$_E(#1)$}}

\newcommand{\procTest}{\procfont{Test}}
\newcommand{\procLR}{\procfont{LR}}
\newcommand{\LRCode}[1]{\underline{\procfont{procedure} \procfont{LR}$(#1)$}}
\newcommand{\sWLR}{\procfont{sW.LR}}
\newcommand{\sWExtract}{\procfont{sW.Extract}}
\newcommand{\goutputs}{\:{\Rightarrow}\:}
\newcommand{\outputs}{\textrm{ outputs }}
\newcommand{\wins}{\textrm{ wins}}
\newcommand{\union}{\;\cup\;}

%\renewcommand{\fbox}[1]{\fbox{#1}}
\newcommand\ch[1]{\textcolor{blue}{#1}}
\newcommand\chbox[1]{\dbox{\ch{#1}}}
\newcommand\del[1]{\textcolor{red}{#1}}
\definecolor{attacker}{rgb}{.9,.9,.9}
\newcommand\att[1]{\colorbox{attacker}{\ensuremath{#1}}}






\subsection*{Policy-Based Signatures} %\cite{  Bellare}.

%\subsubsection{Policy Checker.} A policy checker
%is an NP-relation $P: \bits^* \times \bits^* \to \bits$. The first input is a
%pair $(p, m)$ representing a policy $p \in  \bits^*$ and a message
%$m \in  \bits^*$, while the second input is a witness

\subsubsection{Predicate.}Let $\M$ be a message space and let $P: 2^{\M} \times \M \to \bits$ be
a predicate from sets over $\M$ and a message in
$\M$ to a bit. We say that a message $m \in \M$
is derivable from the set $M \subseteq \M$ if $P(M,m) = 1$.

A functional signature scheme $\FSig$ for a message space $\M$ and a predicate $P: 2^{\M} \times \M \to \bits$ is a tuple of probabilistic, polynomial-time algorithms
$(\FSetup, \FKG, \FSign, \FVer)$ that work as follows

\begin{itemize}
\item $\FSetup(1^\lambda)$: takes a security parameter $\lambda$ and outputs a 
master verification key $\mvk$ and a master secret key $\msk$.

\item $\FKG(\msk, M)$: takes the master secret key $\msk$ and a description $M \subseteq \M$ and it outputs a signing key $\sk_P$ for $P$.

\item $\FSign(\mvk, f, \sk_f, m)$: takes as input a function $f \in \F$, the signing key
$\sk_f$ for that, and a message $m \in  \D_f$, and it outputs $(f(m), \sigma)$ where
$\sigma$ represents a  signature on $f(m)$.

\item $\FVer(\mvk,  m^*, \sigma)$: takes as input the master verification key $\mvk$,
a message $m^* \in \M$ and a signature $\sigma$, and outputs either $1$ (accept)
or $0$ (reject).
\end{itemize}
and satisfy {\em correctness}, {\em unforgeability},  and {\em function privacy}
as described below. 
\begin{itemize}

\item \textbf{Correctness.} A functional signature scheme is correct if the following 
holds with probability 1:
$$\forall f \in \F, \ \forall m \in  \D_f, \ (\msk, \mvk) \gets \FSetup(1^\lambda), \  \sk_f  \gets \FKG(\msk, f), $$
$$  (m^*,\sigma) \gets  \FSign(\mvk, f,  \sk_f, m), \FVer(\mvk, m^*, \sigma) = 1 $$



\item \textbf{Unforgeablity.} A functional signature scheme is unforgeable if for 
every PPT adversary $\A$ there is a negligible function $\epsilon$ such that 
$\Pr[\fsufcma_{\A, \FSig}(\lambda) = 1] \leq \epsilon(\lambda)$ where the experiment 
$\fsufcma_{\A, \FSig}(\lambda)$ is described in the following:\\

\begin{itemize}

\item[{ Key gen:}] The challenger generates $(\msk, \mvk) \gets  \FSetup(1^\lambda)$,
and gives $\mvk$ to $\A$.

\item[{ Queries:}] The adversary is allowed to adaptively query a key generation oracle $\Ok$
and a signing oracle $\Os$, that share a dictionary $D$ indexed by tuples $(f, i) \in \F \times \mathbb{N}$,
whose entries are signing keys. For answering these queries, the challenger proceeds as follows:

\item $\Ok$ $(f, i)$:

\begin{itemize}

\item If $(f, i) \in D$ (i.e., the adversary had already
queried the tuple $(f,i)$), then the challenger replies with the same
key $\sk_f^i$ generated before.

\item Otherwise, generate a new $\sk_f^i \gets \FKG(\msk,f)$, add the entry 
$(f,i) \to \sk_f^i$ in $D$, and return $\sk_f^i$.
 
\end{itemize}

\item $\Os$ $(f, i,m)$: 
\begin{itemize}
\item If there exists an entry for the key $(f, i)$ in $D$, then the challenger
generates a signature on $f(m)$ using this key, i.e., 
$\sigma \gets \FSign(\mvk, f, \sk_f^i, m)$.

\item Otherwise, generate a new key $\sk_f^i \gets \FKG(\msk,f)$, add an
entry $(f,i) \to \sk_f^i$ to $D$, and generate a signature on $f(m)$ using this
key, i.e., $\sigma \gets \FSign(\mvk, f, \sk_f^i, m)$.\\

\end{itemize}

\item[Forgery:] After the adversary is done with its queries, it outputs a pair
$(m^*, \sigma)$, and the experiment outputs 1 iff the following conditions hold
\begin{itemize}
\item $\FVer(\mvk,  m^*, \sigma)=1$.
\item there does not exist $m$ such that $m^* = f(m)$ for any $f$ which was
 sent as a query to the $\Ok$ oracle.
\item there does not exist a pair $(f,m)$ such that $(f,m)$ was a query to the
 $\Os$ oracle and $m^*= f(m)$.
\end{itemize}
\end{itemize}


\item \textbf{Function privacy.} Intuitively, function privacy requires that the
distribution of signatures on a message $m$ that are generated via different
keys $\sk_f$ should be computationally indistinguishable, even given the 
secret keys and master signing key. 
More formally, a functional signature scheme has function privacy if for every
PPT adversary $\A$ there is a negligible function $\nu$ such that
$\Pr[\fspriv_{\A, \FSig}(\lambda) = 1] \leq \nu(\lambda)$ where experiment 
$\fspriv_{\A, \FSig}(\lambda)$ works as follows:
\begin{itemize}
\item The challenger generates a key pair $(\mvk, \msk)  \gets \FSetup(1^\lambda)$
and gives $(\mvk, \msk)$ to $\A$.
\item The adversary chooses a function $f_0$ and receives an (honestly generated)
 secret key $\sk_{f_0} \gets \FKG(\msk, f_0)$.
\item The adversary chooses a second function $f_1$ such that $|f_0|=|f_1|$ (where
 padding can be used if there is a known upper bound) and receives an (honestly 
 generated) secret key $\sk_{f_1} \gets \FKG(\msk, f_1)$.
\item The adversary chooses a pair of values $(m_0, m_1)$ such that $|m_0| = |m_1|$
 and $f_0(m_0) = f_1(m_1)$.
\item The challenger selects a random bit $b \gets \bits$ and computes a signature  
 on the image message $m^* =f_0(m_0) = f_1(m_1)$ using secret key $\sk_{f_b}$,
 and gives the resulting signature $\sigma \gets  \FSign(\sk_{f_b}, m_b)$ to $\A$.
\item The adversary outputs a bit $b'$, and the experiment outputs 1 iff $b' = b$.
\end{itemize} 
\end{itemize}




\subsubsection{P-homomorphic Signature Scheme.} %\cite{AhnBonCam2011}.
\ \\
\paragraph{Derived Messages}
Let $\M$ be a message space and let $P: 2^{\M} \times \M \to \bits$ be
a predicate from sets over $\M$ and a message in
$\M$ to a bit. We say that a message $m \in \M$
is derivable from the set $M \subseteq \M$ if $P(M,m) = 1$. \\


A P-homomorphic Signature Scheme $\PSig$ for a message space $\M$ and predicate
$P: 2^{\M} \times \M \to \bits$ is a tuple of probabilistic, polynomial-time algorithms
$(\PSetup, \PSign, \Derive, \PVer)$ that work as follows
\begin{itemize}
\item $\PSetup(1^\lambda)$: takes a security parameter $\lambda$ and outputs a 
verification key $\pk$, a secret key $\sk$ and a predicate $P$.
We treat the secret key $\sk$ as a signature on the empty tuple $\epsilon \in \M$.
We also assume that $\pk$ is embedded in $\sk$.


\item $\PSign(\sk, m)$: it outputs a fresh signature $\sigma$ on the message $m \in \M$.

\item $\Derive(\pk, m, \lbrace (\sigma_i, m_i) \rbrace)$: the algorithm takes as input the public key, a set of messages $M= \lbrace m_i \rbrace$ and corresponding signatures $\lbrace \sigma_i \rbrace$, a derived message $m$, and produces a new signature $\sigma_m$ or a special symbol $\perp$ to represent failure. 

\item $\PVer(\pk, m, \sigma_m)$: given a public key, message, and purported signature $\sigma$, the algorithm returns $1$
if the signature is valid and $0$ otherwise.
\end{itemize}

A P-homomorphic Signature Scheme $\PSig$ must satisfy {\em correctness}, {\em unforgeability},  and {\em context hiding}
as described below. 
\begin{itemize}

\item \textbf{Correctness.} 
%We require that for all key pairs $(\sk, \pk)$ generated by $\PSetup(1^\lambda)$
%and for all $M \subset \M$ and
%$m \in \M$ we have:
%\begin{itemize}
%
%\item if $P(M,m) = 1$ then $\Derive(\pk, m, (Sign(\sk,M),M)) \neq \perp$
%\item and
%\end{itemize}



\item \textbf{Unforgeablity.} 
\end{itemize}

Consider a classical digital signature scheme $\Sigma$ consists of a triple of algorithms  $\Sigma=(\SignKG,\Sign,\SignVer)$ working as follows:
\begin{description}
  \item[$\SignKG(1^{\secpar})$] the key generation takes as input the security parameter $\secpar$ and returns a pair of keys $(\sk,\vk)$.
  \item[$\Sign(\sk, m)$] on input a signing key $\sk$ and a message $m$, the signing algorithm produces 
  a signature $\sigma'$.
  \item[$\SignVer(\vk, m, \sigma')$]  given a triple $\vk,m,\sigma$ the verification algorithm tests if $\sigma'$ is a valid signature on $m$ with respect to verification key $\vk$.
\end{description} 

We construct a functional signature scheme starting from $\Sigma$ and $\PSig$ as follows:\\

 
\bibliographystyle{alpha}
\bibliography{abbrev3,crypto}  % sigproc.bib is the name of the Bibliography in this case
% You must have a proper ".bib" file
%  and remember to run:
% latex bibtex latex latex
% to resolve all references
%
% ACM needs 'a single self-contained file'!
%
%APPENDICES are optional
%\balancecolumns
\end{document}